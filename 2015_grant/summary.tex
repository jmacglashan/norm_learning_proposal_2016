
\centerline{\Large \bf Representing and Learning Norms:}

\vspace{\down}
\centerline{\large \bf A Key to Human-Machine Collaboration}

\vspace{\up}
\paragraph{Overview}

%We propose to study social preferences and the learning of social norms in artificial agents in an effort to further human-machine collaborations.

We propose to develop a norm representation and norm-learning
algorithms, giving machines with the ability to abide by social norms,
which is required for effective human-machine collaborations.

{\bf Keywords}: reinforcement learning, stochastic games, behavioral experiments, social preferences.

\vspace{\up}
\paragraph{Intellectual Merit}

One vision of a society strewn with ``smart'' devices is
%For society to benefity fully from the ``smart'' devices that pervade our lives, 
machines that help humans in nearly our daily activities, 
ranging from folding laundry to assisting with physical therapy.
%\commenta{folding laundry is the only example i can ever come up with because it is the main thing i need a machine to do for me!!!}
A necessary condition for the success of such future human-machine collaborations 
%in critical applications such as driving 
is productive social interaction between humans and machines.  As
people often expect other people to adhere to \mydef{social norms}
when collaborating, our proposal focuses on developing ways to endow
machines with an ability to learn such norms.
%and tests this ability empirically. 
A social norm is an expectation shared by a group, in which each
member of the group believes others share this same expectation.  For
example, American drivers drive on the right and expect others to do
the same.  Without this shared expectation, transportation and safety
would be greatly impaired.

%One key to productive social interactions are \mydef{social norms}, the focus of this proposal.
%we claim that forming close working relationships requires an ability to represent and learn norms.

A natural computational framework for understanding the interaction of
agents is game theory. However, most successful game-theoretic
learning algorithms to date fall into two
categories---\emph{followers}, which seek a best response to observed
other-agent behavior, and \emph{leaders}, which select a behavior
independent of that which is observed. To represent and learn norms,
participants need to exhibit both of these qualities in an integrated
way. They need to create \mydef{joint plans} that can respond to
observed behavior in the environment. We plan to create and analyze
new norm representations and learning algorithms that lead to
productive joint decision making.

%Our project will make progress towards human-machine collaboration by pursuing three aims:

We propose to study social norms from a computational perspective so
that we can better understand how machines and people can work
together, with the ultimate goal of building machines that we can rely
on as our partners. Specifically, our project will pursue three aims:
%
(1)~developing computational formalisms required for representing
norms in artificial agents; (2)~showing that norms are learnable by
artificial agents in our computational representation by implementing
and evaluating algorithms for norm learning in multi-agent settings;
and (3)~evaluating our computational methods using behavioral
experiments in controlled environments that are specifically designed
to allow for the possibility of human-machine collaboration via the
emergence of social norms.
%demonstrating a proof of concept by applying our norm representation
%and learning algorithms to collaborative tasks that require
%interactions between humans and machines.

\vspace{\up}
\paragraph{Broader Impact}

%The existence of machines that can effectively collaborate with humans and support human decision making would have positive implications for all human-machine interactions, impacting domains ranging from robotic sous chefs to gerontechnological support for aging in place.

%\commenta{Betsy suggests: Internet of things}

The many devices in our world that collect and share data among
the ever-growing ``Internet of Things"
%\url{http://www.itu.int/ITU-T/recommendations/rec.aspx?rec=y.2060}.
cannot yet intelligently coordinate their actions to carry out complex
tasks without extensive human programming and configuration.  As this
collection of devices grows, and begins to include more traditional
robotic actors and more sophisticated software agents, the need for
these entities to not just communicate but work collaboratively is
only going to increase.  We will want and need them to work together
with and without us to accomplish everyday tasks and help us make
better decisions.
%
The range of domains that would be positively impacted by machines
that can collaborate effectively with humans and support human
decision-making is endless, ranging from robotic sous chefs to
gerontechnological support for aging in place, to name two.

%\commentj{Perhaps another impact could be aiding good AI, which is a priority for some afraid of AI. 
%I don't really like this, but figured I'd mention it in case others do.}

In an effort to increase diversity in computer science, we have
pre-selected two graduate students to participate in this
project---one woman, and the other Hispanic.  All students (both
graduate and undergraduate) who join our team will learn the benefits
of collaboration with cognitive psychologists on our team.  Specifically, they will
strengthen their understanding of a number of fields, all of which are
critical to the development of artificial agents that collaborate
effectively with humans: e.g., behavioral economics, cognitive
psychology, reinforcement learning, software engineering, etc..
%
We also plan to integrate our work on this project into Artemis, a
free summer program that introduces rising 9th grade girls to
computational thinking.
%\commentj{Explicit mention that this is a STEM priority?}  
For example, we might have the Artemis girls teach a robot to
collaborate with them on various tasks, such as navigation or object search.
%folding laundry.
%\commenta{insert a different example!!!}
%\commenta{You know, girls should be doing housework.}

Our deliverables include an open-source publicly accessible toolkit for machine norm learning via reinforcement learning. Further, we will build a database of machine-machine, human-machine, and human-human experimental results on norm-learning problems, which can serve as a benchmark for future researchers to build artifical agents that increasingly achieve human-like behavior. 
%
Finally, we expect to publish the results of the proposed research in
top-tier archival, conference proceedings and journals with high
impact factors, and present our work at both computer science and
psychology conferences.

%\paragraph{Collaboration Plan}
%\paragraph{Students' Experiences}
%\paragraph{Unique Opportunity}



\section{Evaluation}

We will evaluate our approach in multiple ways. First, we will test
our interactive learning algorithm with other human players in our
experimental testbed and compare how often and quickly humans are able
to arrive at coordinated behavior with our artificial agent versus
other humans. To test the expressiveness of our algorithm's space of
learnable solutions, we will also evaluate how many different kinds of
strategies can be found when players play with our agent and compare
that to the set of strategies that can be found by humans playing with
each other.  Subjective evaluations from the human participants will
be used to assess whether interacting with the agent feels natural.

For the batch learning algorithm, we will develop new quantitative
metrics to assess behavior similarity so that trajectories produced
from the batch learning can be quantitatively compared to the actual
human participants. Additionally, using our experimental testbed, we
will have two humans interact for a number of sessions and then,
without notification to the users, swap out their partner with an
artificial agent trained from the history of interactions they had
with their human partner. At the end of all interactions, we will ask
the user whether they thought they were playing with the same partner
the whole time or if their partner switched. Their subjective response
to whether they thought their partner was switched at any point will
be compared to their subjective response when their partner is swapped
out with an artificial agent that did not perform any learning and
when their partner is swapped with a different human player.

Finally, we are also interested in norms generalizing to new state
spaces. Using our environmental testbed, we will have participants
play in a sequence of different grids and using the same kinds of
evaluations listed above, test whether our algorithm can safely
generalize to new grids as well as other humans can.


Behavioral economics have explored when people collaborate even when
collaboration is suboptimal from the point of view of a ``selfish''
rational agent.  One explanation relies on other-regarding
preferences, which incorporates the payoff of others into an
individual's preferences. These models fall into three categories:
distributive, reciprocal, and a combination of these. Distributive
models~\cite{fehr1999theory} theorize that other-regarding preferences
arise from an aversion to or preference for inequity among
individuals. Reciprocal models~\cite{berg1995trust} suggest that
individuals consider the payoffs of others because they have
experience with trust and reciprocity, including punishment. More
recent work has brought these two paradigms
together~\cite{charness2002understanding, falk2003nature,
  bolton2000erc}.



\centerline{\Large \bf Socially Rational Artificial Agents:}

\vspace{\down}
\centerline{\large \bf A Key to Human-Machine Collaboration}

\vspace{\up}
\paragraph{Overview}

\emph{Under the assumption that human agents are, perhaps boundedly but
nonetheless ideally, socially rational creatures, we propose to design
and build socially-rational artificial agents that learn through
repeated play, with the aim being that such agents can learn to
collaborate effectively with humans.}

\emph{We propose to study social preferences and the learning of
collaborative behavior in artificial agents in an effort to further
human-machine collaborations.}

\emph{We propose an approach to the design of artificial agents,
by which they interact with humans, simultaneously learning social
preferences and collaborative behaviors that reinforce one another.}

%The ultimate goal of this project is to design artificial agents that
%play well with humans.  We propose to achieve this goal by building
%agents that infer peoples' utility functions, and then learn to
%collaborate appropriately based on these inferences.

{\bf Keywords}: reinforcement learning, stochastic games, behavioral experiments, social preferences.

\vspace{\up}
\paragraph{Intellectual Merit}

A natural computational framework for investigating interactions among
mulitple agents is game theory. Many game-theoretic learning
algorithms to date fall into one of two categories---\emph{followers},
which seek a best response to observed other-agent behavior,
and \emph{leaders}, which select a behavior independent of that which
is observed. To successfully participate in human collective behavior,
artificial agents need to exhibit both of these qualities in an
integrated way. We plan to create and analyze new learning algorithms
that facilitate productive joint decision making among humans and
machines.

%Our project will make progress towards human-machine collaboration by pursuing three aims:

We propose to study collaborative behavior from a computational perspective so
that we can better understand how machines and people can work
together, with the ultimate goal of building machines that we can rely
on as our partners. Specifically, our project will pursue three aims:
%
(1)~developing computational formalisms required for representing
social preferences in artificial agents; (2)~showing that
collaborative behavior is learnable by artificial agents in our
computational representation by implementing and evaluating
reinforcement learning algorithms in multi-agent settings; and
(3)~evaluating our computational methods using behavioral experiments
in controlled environments that are specifically designed to allow for
the possibility of human-machine collaboration.

\commenta{MUST WRITE ABOUT NEW IRL TECHNOLOGY!!!}

%demonstrating a proof of concept by applying our social preference representation and
%learning algorithms to collaborative tasks that require interactions between humans and machines.

\vspace{\up}
\paragraph{Broader Impact}

One vision of a society strewn with ``smart'' devices is
machines that help humans in nearly our daily activities, 
ranging from folding laundry to assisting with physical therapy.
%\commenta{folding laundry is the only example i can ever come up with because it is the main thing i need a machine to do for me!!!}
A necessary condition for the success of such future human-machine collaborations 
%in critical applications such as driving 
is productive social interaction between humans and machines.  
%
The range of domains that would be positively impacted by machines
that can collaborate effectively with humans and support human
decision-making is endless, ranging from robotic sous chefs to
gerontechnological support for aging in place, to name two.

\commentj{Perhaps another impact could be aiding good AI, which is a priority for some afraid of AI.}

In an effort to increase diversity in computer science, we have
pre-selected two graduate students to participate in this
project---one woman, and the other Hispanic.  All students (both
graduate and undergraduate) who join our team will learn the benefits
of collaboration with cognitive psychologists on our team.  Specifically, they will
strengthen their understanding of a number of fields, all of which are
critical to the development of artificial agents that collaborate
effectively with humans: e.g., behavioral economics, cognitive
psychology, reinforcement learning, software engineering, etc..
%
We also plan to integrate our work on this project into Artemis, a
free summer program that introduces rising 9th grade girls to
computational thinking.
%\commentj{Explicit mention that this is a STEM priority?}  
For example, we might have the Artemis girls teach a robot to
collaborate with them on various tasks, such as navigation or object search.
%folding laundry.
%\commenta{insert a different example!!!}
%\commenta{You know, girls should be doing housework.}

Our deliverables include an open-source publicly accessible toolkit
for implementing human-machine collaborative-learning tasks via
reinforcement learning. Further, we will maintain a database of
machine-machine, human-machine, and human-human experimental results
built using this toolkit, which can serve as a benchmark for future
researchers who also seek to build artifical agents that increasingly
achieve human-like behavior.
%
Finally, we expect to publish the results of the proposed research in
top-tier archival, conference proceedings and journals with high
impact factors, and to present our work at innovative, non-archival
workshops (e.g., the AAAI symposia).


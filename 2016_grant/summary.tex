
\centerline{\Large \bf Socially Rational Artificial Agents}

\vspace{\down}
\centerline{\large \bf A Key to Human-Machine Collaboration}

\vspace{\up}
\paragraph{Overview}

\emph{Under the assumption that humans are, perhaps boundedly
  but nonetheless ideally, socially rational creatures, we propose to
  design and build socially rational artificial agents that learn via
  repeated interactions, with the aim being for such agents to
  collaborate effectively with humans.}

\comment{
\emph{We propose to develop collaborative learning algorithms for 
artificial agents in an effort to foster human-machine
collaborations.}

\emph{We propose an approach to the design of artificial agents,
by which they interact with humans, simultaneously learning social
preferences and collaborative behaviors that reinforce one another.}
}

%The ultimate goal of this project is to design artificial agents that
%play well with humans.  We propose to achieve this goal by building
%agents that infer peoples' utility functions, and then learn to
%collaborate appropriately based on these inferences.

{\bf Keywords}: reinforcement learning, stochastic games, behavioral experiments, social preferences.

\vspace{\up}
\paragraph{Intellectual Merit}

Much of human life occurs in contexts where people must coordinate
their actions with those of others.  From party planning to space
exploration to grant-proposal evaluation, groups of people have
accomplished great things by reasoning as a team and engaging in
jointly intentional behavior.  Indeed, some have argued that, because
most other animals lack the capacity to work adaptively as a cohesive
unit across many domains, team reasoning may be the hallmark of human
sociality.

We call optimal decision making, when agents can hold social
preferences and employ team reasoning as appropriate, \mydef{socially
rational behavior}.  Socially rational agents attempt to optimize a
social utility function (i.e., a representation of social
preferences), which is sufficiently rich to incorporate perceived
societal benefits.  We assume that social utilities can be broken down
into two components---an objective component, which is usually a
direct function of the rules of interaction, and a subjective
component, which captures notions of distributivity and reciprocity.

Given these assumptions, we propose an iterative computational model
in which socially rational artificial agents construct social
preferences from observed histories of repeated interactions with
other, potentially human, agents, and then decide how to optimize
them.  We contend that socially rational behavior, in which agents can
jointly optimize a learned social utility function that reflects
constructed social preferences, is a promising avenue for
orchestrating collaborations between humans and machines.

Central to our computational model is the notion of inverse
reinforcement-learning (IRL), whereby an agent is shown demonstrations
of behavior and based on which it infers utilities that motivate that
behavior.  Nearly all existing IRL algorithms to date assume the
demonstrations are generated by an expert.  In our setting, in
contrast, the demonstrations will be past interactions among agents
who are not necessarily skilled at the task at hand, but, rather, are learning.
%on the contrary, (hopefully) learning to collaborate.  
Consequently, we are proposing to develop
%novel technical tool for learning social utility functions is
new IRL technology that learns social utility functions from
%(which is presently in its infancy)
both bad and good examples of behavior.  This technology will enable
humans to give agents both positive and negative feedback while they
are learning to collaborate.

In reality, when an agent arrives on the playing field, it does not
know whether the other agents it faces err on the side of being social
(team reasoners) or selfish (best-repliers).  In the context of the other agents' behavior,
it behooves a socially rational agent to decide upon a strategy
for itself---social or selfish?
%
We will also develop and evaluate algorithms that classify
environments as social or selfish, and adopt the corresponding stance.

%We will also develop and evaluate a new inverse reinforcement-learning
%algorithm that can construct social utility functions by analyzing
%the successes and failures of past interactions.


\vspace{\up}
\paragraph{Broader Impact}
%A wide variety of human--machine interaction problems would be
%positively impacted by the technology we are proposing.
%
This project is part of Brown University's Humanity Centered
Robotics Initiative (HCRI) and its ongoing efforts to design robotic
systems that interact with people and support independent living tasks
(e.g., gerontechnological support for aging in place).  For an elderly
person to trust and collaborate on tasks with a machine effectively,
the machine must act in a manner that the elderly person expects.  Our
proposed project is foundational for these
important applications.

To engage a wider group in these efforts, we will create an
undergraduate course called ``Social autonomous driving'' to be
offered as part of Brown's new robotics course sequence.  Students
will develop robot cars that drive around a test environment, making
sure they interact smoothly with other robotic and remote-controlled
cars.  We also plan to integrate our work on this project into
Artemis, a free summer program that introduces rising 9th grade girls
to computational thinking by having the Artemis girls teach a robot to
collaborate with them directly on routine tasks.

Our deliverables include an open-source publicly accessible toolkit
for implementing human-machine collaborative-learning tasks via
reinforcement learning. Further, we will maintain a database of
machine--machine, human--machine, and human--human interactions, which can serve as a benchmark for future
researchers who also seek to build artificial agents that increasingly
achieve human-like behavior.
%
Finally, we expect to publish the results of the proposed research in
top-tier archival conference proceedings and journals with high
impact factors, and to present our work at innovative, non-archival
workshops (e.g., the AAAI symposia).


Our data management plan is designed to meet the letter and spirit of
NSF data-management policy as described in the following documents:
\begin{itemize}
\item NSF 11-1 Grant Proposal Guide: Sec. II.C.2.J Special Information
   and Supplementary Documentation.
\item NSF AAG11-001 Award and Admin. Guide Section VI.D.4.: Dissemination
   and Sharing of Research Results.
\end{itemize}

\subsection*{Expected Data and Other Products}

The expected data generated by this research includes open-source
software libraries, vehicle-navigation experiments datasets, and
performance benchmarks. The software will be collected and stored in a
GIT repository.  The PIs will provide open-source implementations of
parts of the software dealing with decision making and learning.

A common GIT file repository will be employed for code development and
data set storage.

For each human experiment, the collected data will include:
\begin{itemize}
\item definition files with a description of the study materials used.
\item trajectories with time-stamped states, controls, costs
\end{itemize}

These materials will be completely anonymized with no personally
identifying attributes.

For machine experiments, we will build on the BURLAP library and will
make our extensions available via GIT.

\subsection*{Period of Data Retention}

Data will be retained for a minimum of five years after completion of
the grant.

\subsection*{Data Formats}

Human trajectory data will be stored in ASCII files along with a
javasccript program for interpreting/visualizing them.

Other data will be stored in either ASCII or binary form.

\subsection*{Data Access and Sharing}

The software and collected log files will be made publicly available
at the time of the first release of an associated publication. The
data will be posted on a Brown-maintained web server, then moved to a
Brown-maintained archive at the end of the project. All data is
network accessible and routinely backed up onto physical drives. Upon
substantial increase in data size, the PIs will consider
data-management services such as available from Brown or Google
Cloud. The PIs will specify conditions for data use, designate the
original source of data, and associate it with the NSF grant number.

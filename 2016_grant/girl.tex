
\subsection{Generalized Inverse Reinforcement Learning}
\label{sec:girl}

To complement the new computational model we put forth, the new
technology that we plan to develop as part of this proposal is a new
model of IRL, and accompanying algorithms, which allows for both
positive and negative (not necessarily binary) labels, and, moreover,
which also allows for partial feedback, meaning subtrajectories (i.e.,
options~\cite{sutton99}) can be ascribed such labels.

To this end, in this section, we propose a novel variant of the
inverse reinforcement learning problem, which we call
\mydef{Generalized IRL} (GIRL).  Wherease the input to standard IRL is
a data set of trajectories
%(a sequence of state-action pairs)
and the output, a reward function, the input to a GIRL problem, is a
data set of \emph{labelled\/} trajectories. 
%(The output is still a reward function.)  
In its most basic formulation,%
\footnote{Labels need not be binary; they can be continuous-valued.}
each label is either positive ($+1$) or negative ($-1$), indicating
whether the associated trajectory is a good example of some desired
behavior, or a bad example of some desired behavior.  The goal is to
recover a reward function that motivates the desired behavior.

The plate diagram that describes our GIRL problem formulation is shown
in Figure~\ref{fig:girl}.  In this model, we interpret the label ($L$)
associated with a trajectory of size $N$ to be some random function of
what the evaluator thought about each of the action selections ($A$)
exhibited at each of the states ($S$) in the trajectory.  However,
these step-wise evaluations ($X$) are unobserved in the data.

\begin{wrapfigure}{l}{0.366\textwidth}
\begin{tikzpicture}[x=1.7cm,y=1.8cm]
   \node[latent]                   (R)      {$\bm{\theta}$} ; %
   \node[latent, right=of R]                   (X)      {$X$} ; %
   \node[obs, below=of X]                      (L)      {$L$} ; %
   \node[obs, above=of X]          (S)      {$S$} ; %
   \node[obs, right=of S]          (A)      {$A$} ; %
   
   \factor[above=of L]{L-f}{right:Sigmoid}{}{} ;

  \edge {S}{X} ;
  \edge {A}{X} ;
  \edge {R}{X} ;
  \factoredge {X}{L-f}{L} ;

   \plate {traj}{
   		(S) (A) (X)
   } {$N$} ;%

   \plate {data}{
   		(traj)(L-f)(L)
   } {$M$} ;

\end{tikzpicture}
\caption{Generalized IRL}
\label{fig:girl}
\end{wrapfigure}

For illustrative purposes, we adopt the probabilistic maximum
likelihood problem formulation, in which we seek a reward function
that maximizes the likelihood of the data, but Bayesian reasoning
could likewise be applied here.
%
To find the reward function that maximizes the likelihood of the data,
we first define a probability model for the system that is represented
in Figure~\ref{fig:girl}.
%
We model the probability that an action is evaluated as good or not as
proportional to its selection probability according a softmax policy
computed for the reward function with parameters
$\bm{\theta}$. Specifically:
%
\begin{align}
\Pr(x_i = +1 \mid s, a, \theta) &= \pi(s, a \mid \bm{\theta}) \\
\Pr(x_i = -1 \mid s, a, \theta) &= 1 - \pi(s, a \mid \bm{\theta}),
\end{align}

\noindent
Here $\pi(s, a \mid \bm{\theta})$ is the softmax policy over $Q$-values,
assuming a reward function parameterized by $\bm{\theta}$.

\comment{
%%% DELETED B/C OF SPACE CONSTRAINTS !!!
\begin{equation}
\pi(s, a \mid \bm{\theta}) = \frac{e^{\beta Q(s,a \mid \bm{\theta})}}{\sum_{a'}e^{\beta Q(s,a' \mid \bm{\theta})}},
\end{equation}

\noindent
$\beta$ is a selectable parameter, and $Q(s,a \mid \bm{\theta})$ is the
$Q$-value computed for the reward function parameterized by
$\bm{\theta}$.
}

For the probability distribution of $L$, given the sequence of $N$
step-wise labels, we would like a distribution that has the property
that as more step-wise labels are positive, the probability of a
positive trajectory label increases (and vice versa). Although there
are many possible distributions that satisfy this property, for
concreteness, we choose the sigmoid function. That is,
%
\begin{align}
\Pr(L = +1 \mid X_1, ..., X_n) &= \frac{1}{1 + e^{-\phi \sum_i^N X_i}} \\
\Pr(L = -1 \mid X_1, ... ,X_n) &= 1 - \Pr(L = +1 \mid X_1, ..., X_n),
\end{align}

\noindent
Here $\phi$ is a parameter that controls how readily step-wise labels
influence the trajectory labels.  For example, when $\phi = 0$,
trajectory labels are assigned uniformly at random, without any regard
for of step-wise labels.  As $\phi \rightarrow \infty$, the sigmoid
converges to a step function in which a trajectory containing even one
more positive step-wise label than negative will deterministically
lead to a positive trajectory label (and vice versa).


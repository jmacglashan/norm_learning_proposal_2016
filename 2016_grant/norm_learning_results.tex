%Larger RHIRL horizon can allow for more temporally abstract equilibria to be encoded 
%ex: "pass on the left" may take multiple steps to coordinate and give a reward when agents pass each other on the left

\vspace{\up}
\paragraph{Preliminary Results}
\label{sec:results}

Preliminary experiments suggest that both our batch-learning and interactive
collaborative learning algorithms can produce reasonable results in
grid games. For batch learning, we hard code examples of agents
following different equilibrium strategies, and show that our
algorithm is able to recover the appropriate equilibrium when it is
trained on demonstrations of behavior following that equilibrium. We
also tested the agents on similar grid games after learning to show
that the learned behavior can generalize to other similar situations.
For interactive learning, we show that agents can very quickly
%(typically after one round) 
find an equilibrium, and that restarting learning from scratch can
result in them finding a different one.

For both sets of experiments, we examine behavior in Hallway.  The
social utility function consists of a total welfare team 
function and a linear family of bias functions. The bias functions
operate on 50 state, joint action binary features. The first 25
features are an indicator variable for selecting one of the 25 joint
actions when the agents are in {\em conflict\/} and the latter 25
features are indicator functions for selecting one of the 25 joint
actions when the agents are not in conflict. We define the agents to be
in conflict when they are in the same row of the grid and each is
closer to their opponent's goal than they are to their
own. Additionally, when one of the agents is one step away from their
goal, none of the features activate, thereby preventing premature
termination on the joint task. Since these features directly encode
social utilities for joint actions, rather than higher-level goals, when
training we use RHIRL with a horizon of 1. We do not expect this
set of features to be complete or capable of capturing all the
possible behaviors that might emerge in a grid game. So, an important
area in this project is to identify features that are both expressive
and generalize well. Nevertheless, the features used here provide a
proof of concept for our approach.
%which can later be extended to more expressive features.

%\subsection{Batch Training Results}

\begin{figure}
    \centering
    \begin{subfigure}[b]{0.275\textwidth}
        \includegraphics[width=\textwidth]{../2015_grant/figures/batch1}
        \caption{}
        \label{fig:batch1}
    \end{subfigure}
\qquad
    ~ %add desired spacing between images, e. g. ~, \quad, \qquad, \hfill etc. 
      %(or a blank line to force the subfigure onto a new line)
    \begin{subfigure}[b]{0.275\textwidth}
        \includegraphics[width=\textwidth]{../2015_grant/figures/batch2}
        \caption{}
        \label{fig:batch2}
    \end{subfigure}
\qquad
    ~ %add desired spacing between images, e. g. ~, \quad, \qquad, \hfill etc. 
    %(or a blank line to force the subfigure onto a new line)
    \begin{subfigure}[b]{0.275\textwidth}
        \includegraphics[width=\textwidth]{../2015_grant/figures/batch3}
        \caption{}
        \label{fig:batch3}
    \end{subfigure}
    \caption{The three different equilibria the batch-learning algorithm trained on in Hallway. Colored arrows indicate the path of the player of the same color. A small circle indicates a wait action.}\label{fig:batchRes}
\end{figure}

We tested our batch-learning algorithm on datasets in which
agents followed three different equilibrium strategies. 
%All three equilibrium behaviors are shown in \ref{fig:batchRes}. 
The first (Figure~\ref{fig:batch1}) shows Orange going
north and then traveling along the top row of the grid, while Blue goes south and then travels along the bottom row. The second
(Figure~\ref{fig:batch2}) shows Blue going north and then
traveling along the top row while Orange waits, then goes
east, and then waits again for Blue to catch up so they can enter
their goals together.  The third (Figure~\ref{fig:batch3}) has 
Orange go south and then traveling along the bottow row, while
Blue immediately goes west toward its goal and then waits
two steps.
%for orange to catch up so they can enter their goals together.

For each of these three equilibria, we generated five demonstrations,
two of which described the behavior exactly, and the other three of
which contained slight deviations (for example, moving back to the
center row before reaching the end). We then separately trained our
batch-learning algorithm on each set of demonstrations. In all cases, the
learned social utility function motivated behavior that consistently
replicated the data on which it was trained. Additionally, we planned
using the learned social utility function in a larger $7\times 3$ hallway grid
game, and, in all cases, obtained the corresponding equilibrium behavior.

%\subsection{Interactive Training}
%\jmnote{I will also try to run an experiment in which one player is initially biased to show that the other agent adapts to them.}

\begin{figure}
    \centering
    \begin{subfigure}[b]{0.18\textwidth}
        \includegraphics[width=\textwidth]{../2015_grant/figures/interactive1}
        \caption{}
        \label{fig:inter1}
    \end{subfigure}
    ~ %add desired spacing between images, e. g. ~, \quad, \qquad, \hfill etc. 
      %(or a blank line to force the subfigure onto a new line)
    \begin{subfigure}[b]{0.18\textwidth}
        \includegraphics[width=\textwidth]{../2015_grant/figures/interactive2}
        \caption{}
        \label{fig:inter2}
    \end{subfigure}
    ~ %add desired spacing between images, e. g. ~, \quad, \qquad, \hfill etc. 
    %(or a blank line to force the subfigure onto a new line)
    \begin{subfigure}[b]{0.18\textwidth}
        \includegraphics[width=\textwidth]{../2015_grant/figures/interactive3}
        \caption{}
        \label{fig:inter3}
    \end{subfigure}
	~ %add desired spacing between images, e. g. ~, \quad, \qquad, \hfill etc. 
    %(or a blank line to force the subfigure onto a new line)
    \begin{subfigure}[b]{0.18\textwidth}
        \includegraphics[width=\textwidth]{../2015_grant/figures/interactive4}
        \caption{}
        \label{fig:inter4}
    \end{subfigure}
    ~ %add desired spacing between images, e. g. ~, \quad, \qquad, \hfill etc. 
    %(or a blank line to force the subfigure onto a new line)
    \begin{subfigure}[b]{0.18\textwidth}
        \includegraphics[width=\textwidth]{../2015_grant/figures/interactive5}
        \caption{}
        \label{fig:inter5}
    \end{subfigure}
    \caption{The learning results from each of the 5 interactive matches (a-e). The top image for a match is the first round of interaction of the agents. The bottom image for a match is the learned behavior. For matches b-e, the last move of an agent that reaches their goal sooner may be to wait for two steps, or to move north and/or south and back again; only one example is illustrated.}\label{fig:interRes}
\end{figure}

In our interactive learning experiment,
%(see Figure~\ref{fig:interRes}), 
we played two interactive learning agents against one another.  The
agents played five matches, each one consisting of five rounds.  Each
match began fresh, without any knowledge of previous matches, to see
whether different equilibria could emerge.  The social utility function was
initialized to the total welfare joint policy.  In each match, the
agents very quickly (typically after the first round) converged to an
equilibrium that persisted for the rest of the match.  However, in each
match, a different equilibrium was learned, showing that agents employing our
algorithm quickly adapt to the particular shared experience they have
with other agents.
%, and then maintain the learned equilibrium behavior.
% zzz can we quantify how much faster it is than straight Q-learning?

Figure~\ref{fig:interRes} shows the results of learning in each of the
five matches (with the top image showing the behavior in the first
round and the bottom image showing the learned behavior). While
different equilibria emerged in different matches, once learned, that
equilibrium was consistently employed in all subsequent rounds.  Of
particular interest are the results in the fifth match
(Figure~\ref{fig:inter5}). In the first round of this match, both
players wait (a suboptimal joint action), and Blue then moves in an
odd looping motion. Nevertheless, the agents learn from the experience,
ultimately finding collaborative behavior that works well for them both.

\comment{
Although any off-the-shelf IRL algorithm would be applicable here, in
these experiments, we use receding-horizon, maximum-likelihood IRL.
In future work, we plan to use (receding-horizon) \mydef{feedback IRL}
(see Section~\ref{}) instead.
}


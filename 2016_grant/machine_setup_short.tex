
\vspace{\up}
\paragraph{Simulation Experiments}
\label{sec:qlearning}

We carried out a set of simulation experiments
with \Q-learning~\cite{Watkins92} in the grid games presented.
%
% We denote the outcome of a round using a pair of letters, where {\bf G} 
% means the agent reached the goal and {\bf N} means the agent did not
% reach the goal. The first letter in the pair represents the agent's
% own outcome and the second represents its opponent's outcome. For
% example, {\bf GN} is used to denote that the agent reaches its goal
% while its opponent does not.
As two \Q-learning algorithms are not guaranteed to converge in self
play, we arbitrarily stopped the learning after 30,000 rounds, and
checked the strategies learned.  In spite of \Q-learning not
explicitly seeking outcomes with high social welfare, it very reliably
identified collaborative strategies in which both agents reached their
goals.
% leading to {\bf GG} outcomes.


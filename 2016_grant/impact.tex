
\section{Broader Impacts}

Recent trends in computer science and artificial intelligence are
moving us toward increasing dependence on human--computer
collaborative systems in which people and software each make decisions
that impact one another. Some fields, such as human--computer
interaction (HCI), focus on systems in which a human being is
primarily in control and the computer's decisions are assessed in
terms of the positive impact they have on the user. An interesting
recent example is Centaur chess
\url{http://www.huffingtonpost.com/mike-cassidy/centaur-chess-shows-power_b_6383606.html}
in which a human and machine team up to play on the same side in
chess, both making decisions, but with the machine acting as an
advisor and the human deciding which moves to actually make. Other
fields, like human computation and crowd sourcing, combine human and
machine expertise with the machine being the ultimate arbiter of
behavior and a group of human beings acting as lower level
computational components.

True collaboration, however, does not start with one participant being
assigned to a leadership role. Instead, the various agents need to
dynamically negotiate their roles and jockey for position, discovering
when and how to trust each other to move forward. A concrete example
is in the context of self-driving cars. Cars share the road with each
other and must carefully choose when to be responsive to other
vehicles and when to assert themselves to create a situation that
benefits them. Doing so makes a significant difference in the driving's
effectiveness~\cite{cunningham2015mpdm}. A related problem arises when
a self-driving wheelchair attempts to move through a group of
pedestrians~\cite{kim2016socially}. More generally, robots that
interact with people in the physical world need to navigate the
complex give-and-take of establishing mutually beneficial behaviors
where possible.

%%%

\commenta{2015:}
The domains that would be positively impacted by machines
that can collaborate effectively with humans is endless, such as robotic sous chefs and
gerontechnological support for aging in place.
%
In particular, the results of this project synergize with the main efforts of 
Brown University's Humanity Centered Robotics Initiative (HCRI), of
which co-PI Littman is co-director. Specifically, within HCRI there
are ongoing efforts to design robotic systems that support independent
living tasks. For an elderly person to trust and collaborate on tasks with a machine effectively, the machine must act in a manner that the elderly person expects. This project lies the foundation for these important applications.

In an effort to increase diversity in computer science, we have
pre-selected two graduate students to participate in this
project---one woman, and the other Hispanic.  All students (both
graduate and undergraduate) who join our team will benefit from collaborating with the cognitive psychologists on our team.
Specifically, they will strengthen their understanding of a number of
fields, all of which are critical to the development of artificial
agents that collaborate effectively with humans: e.g., behavioral
economics, cognitive psychology, reinforcement learning, and software
engineering.

``One of the things that [President Obama] really strongly believe[s]
in is that we need to have more girls interested in math, science, and
engineering.''  To that end, we also plan to integrate our work on
this project into Artemis, a free summer program run at Brown and
directed by PI Greenwald that introduces rising 9th grade girls to
computational thinking.
%\commentj{Explicit mention that this is a STEM priority?}  
For example, we might have the Artemis girls teach a robot to
collaborate with them on various tasks, such as finding and navigating
to common household objects.
%folding laundry.
%\commenta{insert a different example!!!}
%\commenta{You know, girls should be doing housework.} 
% JLA: yeah, i was concerned about the folding laundry example for that reason... 

Our deliverables include an open-source publicly accessible toolkit for agents to learn collaborative behavior via reinforcement learning. Further, we will build a database of machine-machine, human-machine, and human-human experimental results on collaborative-learning problems, which can serve as a benchmark for future researchers to build artifical agents that increasingly achieve human-like behavior. 
%
Finally, we expect to publish the results of the proposed research in
top-tier archival, conference proceedings and journals with high
impact factors, and present our work at both computer science and
psychology conferences.





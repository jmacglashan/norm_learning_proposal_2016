
\section{Broader Impacts}

Recent trends in computer science and artificial intelligence are
moving us toward increasing dependence on human--computer
collaborative systems in which people and software make decisions that
impact one another.  Some fields, such as human--computer interaction
(HCI), focus on systems in which a human being is primarily in
control, and the computer's decisions are assessed in terms of the
positive impact they have on the user.  An interesting recent example
is Centaur
chess\footnote{\url{http://www.huffingtonpost.com/mike-cassidy/centaur-chess-shows-power_b_6383606.html}}
in which a human and machine team up to play on the same side in
chess, both making decisions, but with the machine acting as an
advisor and the human deciding which moves to actually make.  Other
fields, like crowd sourcing human computation~\cite{von2009human},
combine human and machine expertise such that the machine is the
ultimate arbiter of behavior and a group of human beings act as lower
level computational components.

True collaboration, however, does not start with one participant being
assigned to a leadership role.  Instead, the various agents need to
dynamically negotiate their roles and jockey for position, discovering
when and how to trust each other to move forward.  A concrete example
is in the context of self-driving cars.  Cars share the road with each
other and must carefully choose when to be responsive to other
vehicles and when to assert themselves to create a situation that
benefits them.  Doing so makes a significant difference in the
driving's effectiveness~\cite{cunningham2015mpdm}.  A related problem
arises when a self-driving wheelchair attempts to move through a group
of pedestrians~\cite{kim2016socially}.  More generally, robots that
interact with people in the physical world need to navigate the
complex give-and-take of establishing mutually beneficial behaviors
where possible.

Indeed, a wide variety of human--machine interaction problems would be
positively impacted by the technology we are proposing.
%
As such, the project is synergistic with Brown University's Humanity
Centered Robotics Initiative (HCRI), of which co-PI Littman is
co-director.  Specifically, within HCRI, there are ongoing efforts to
design robotic systems that interact with people and support
independent living tasks (e.g., gerontechnological support for aging
in place and robotic sous chefs).  For an elderly person (or chef) to
trust and collaborate on tasks with a machine effectively, the machine
must act in a manner that the elderly person (or chef) expects.  This
project creates the foundation for these important applications.

The most direct application of our ideas would be in scenarios in
which a robot is sharing a space with bystanders (human or robotic)
who do not necessarily share its goals---a personal grocery-store
assistant, a socially aware vacuum, an automated hospital linen cart,
a package delivery drone.  To further develop our ideas in a physical
environment like these, and moreover in embodied robots,
% in year 3, team up with George and/or Stefanie. by then, there will
% be robots in the department that can both move and move things around
we plan to build a ``go fetch'' application on a mobile-manipulation
platform that is scheduled for deployment in the Brown Computer
Science department in Spring 2017.  To be successful in this capacity,
a robot will need to be able to represent, learn, and apply
collaborative behavior in (at least) three separate spheres:
%
%\begin{itemize}
(1)~navigating smoothly through the atrium and hallways, finding
  appropriate ways to avoid collisions, and adopting regularities that
  make it easy for people to coordinate their own movements with it;
%
(2)~riding in the elevator, which brings with it some related but
  distinct expectations for coordinating (moving in and out of the
  doors, standing inside, asking for help with the buttons, deciding
  what to do when obstructed, etc.);
%
(3)~understanding how to best signal a handoff of the object it is
  fetching, by interrupting people without being too timid or too
  boorish.
%\end{itemize}

%This task, is within our technological limitations (unlike wheelchairs or actual cars) 
%and would definitely result in some impact if our algorithms were able to help.

Interest in robotics and machine learning is growing within the Brown
CS department.  Presently, our faculty is designing a new robotics
introductory course sequence that includes classes on basic robotics
algorithms, hardware design, and human-centered evaluation of robotics
systems.  Inspired by the ideas in this proposal, we plan to
contribute to this effort a new intermediate robotics course for
undergraduates called ``Social autonomous driving.''  As a test
platform, we will use MIT's Duckietown
platform\footnote{\url{http://duckietown.mit.edu/}, which we have installed at Brown}.
Using this platform as a test environment, students will
develop robotic drivers.  The main emphasis will be on making sure
those drivers interact smoothly with other robotic and
remote-controlled cars.  The latter will be achieved by endowing the
robots with socially rational capabilities, thereby enabling them to
adapt to local ``customs'' and driving styles (such as the
``Pittsburgh left'' or the ``Boston rotary'').

%%%

% In an effort to increase diversity in computer science, we have
% pre-selected two graduate students to participate in this
% project---one woman, and the other Hispanic.  All students (both
% graduate and undergraduate) who join our team will benefit from
% collaborating with the cognitive psychologists on our team.
% Specifically, they will strengthen their understanding of a number of
% fields, all of which are critical to the development of artificial
% agents that collaborate effectively with humans: e.g., behavioral
% economics, cognitive psychology, reinforcement learning, and software
% engineering.

The co-PIs have a strong record of mentoring students from
underrepresented groups, and intend to continue pursuing these efforts
in the context of this proposal.  In particular, we plan to integrate
our research on human--machine collaborations into Artemis, a free
summer program directed by co-PI Greenwald that introduces rising 9th
grade girls to computational thinking.  For example, we might have the
Artemis girls teach a robot to collaborate with them on routine tasks,
such as navigation or object search.

%folding laundry.
%\commenta{insert a different example!!!}
%\commenta{You know, girls should be doing housework.} 
% JLA: yeah, i was concerned about the folding laundry example for that reason... 

% Our deliverables include an open-source publicly accessible toolkit
% for agents to learn collaborative behavior via reinforcement
% learning. Further, we will build a database of machine-machine,
% human-machine, and human-human experimental results on
% collaborative-learning problems, which can serve as a benchmark for
% future researchers to build artifical agents that increasingly achieve
% human-like behavior.

Finally, we expect to publish the results of the proposed research in
top-tier archival, conference proceedings and journals with high
impact factors, and to present our work at innovative, non-archival
workshops (e.g., the AAAI symposia).



\vspace{\up}
\paragraph{Evaluation}

%Before scaling to embodied robots
To begin, we will evaluate how our socially-rational agents fare in
collaborative learning tasks using an experimental platform that
simulates interactions between humans and artificial agents.
Specifically, we will compare how often and how quickly humans are
able to arrive at coordinated plans among themselves, versus with
socially-rational agents.  We will also assess the quality of the
plans that emerge under these varying conditions.  To test the
expressiveness of the space of learnable solutions afforded by our
approach, we will also measure tendencies among humans to converge to
certain types of plans (e.g., trading off simplicity for optimality),
and whether those types persist when there are agents in the mix.
Finally, subjective evaluations will be used to assess whether humans
find interacting with agents ``natural.''

For the batch learning algorithm, we will develop new quantitative
metrics to assess behavioral similarity so that trajectories produced
by our batch learners can be quantitatively compared to trace data
produced by human participants.  Additionally, we plan to design a
sort of ``Turing test'' for our problem domain, whereby we will have
two humans interact with each other for a number of rounds and then,
without notification, we will fork those interactions into two, where
each human is now playing with a socially-rational agent that trained
on the history of interactions between the two humans.  At the end of
all interactions, we will ask the humans whether they noticed anything
strange partway through the interactions, such as a change of partner.
Their subjective responses about whether they thought their partner
switched at any point will be compared across a variety of treatments,
such as: (1)~no change; (2)~swapping their partner for a
socially-rational agent as already mentioned, and one that learned by
observing other human--human interactions; (3)~swapping their partner
for another human who observed the previous interactions, and one who
is experienced at the task, but not with the given partner.

We are particularly interested in designing flexible agents that
generalize their behavior at least as well as humans do to new
environments.  We will evaluate generalization by first having users
learn in one environment.  Afterwards, they will be exposed to a new
environment where their previous behavior is no longer directly
applicable (e.g., because a wall was inserted along the path that an
agent would normally take), but a more ``general'' behavior can be
transferred (e.g., I take the high road; he takes the low, regardless
of the presence or absence of obstacles).  Then, using evaluations
similar to those listed above, we will test whether our
socially-rational agents can generalize to new games as well as other
humans can.

% \commenta{do we want to say something about generalizing to other players? maybe humans are better at this than our agents will be?} ML: No, I think the notion of generalizing rapidly to new users is a distinct learning problem.

\commenta{!!!}
Throughout the project, we will investigate how to design social
utility functions that generalize across environments. We we will
start by examining standard feature selection methods used in existing
reinforcement-learning and IRL research (e.g., linear models, neural
networks,
etc.)~\cite{diuk2009adaptive,kolter2009regularization,li2009reinforcement,parr2008analysis},
and build on them, as necessary.

In year three, we will test our approach in a real-world robot--human
collaboration domain.  Even if (as we expect), our simulated worlds
show that our agents collaborate well with people, the capabilities
(or lack thereof) of robots may inhibit the effectiveness of our
approach (e.g., robots might move slower than people expect and/or may
misperceive important details).  We will evaluate our algorithmic
framework in real-world scenarios by comparing its performance at
controlling a robot to another human controlling the robot via
teleoperation.  We expect this approach to raise novel research
questions appropriate to real-world problems.
%Amy: sounds too generic, i think. we always want to develop more efficient implementations, don't we?
%%For example, computation time may be a limiting factor to the success
%%of our machine agents, and if so, we will focus our efforts on
%%developing more efficient implementations.

% OLD STUFF
%
% Evaluation metrics that capture trajectory similarity across diverse
% games are key for assessing performance, and will drive the
% development of our learning algorithms towards more beneficial
% behavior.
% 
% We will also develop ways for people to also assess how natural it is
% to interact with our artificial agents. For example, we will
% investigate whether having two humans interact for awhile, and then,
% without notifying them, swapping their partner with an artificial
% agent trained from the interaction history results in a noticeable
% change in behavior. Participants would be asked if they noticed a
% change, and behavior would be compared between a variety of cases, 
% such as: no swap; or swapping in a trained agent, an untrained agent, 
% a human who witnessed the history of interactions, a human who did not.
%
% We will evaluate generalization by first having users learn in one
% grid. Afterwards, they will be exposed to a new grid where their
% previous behavior is no longer directly applicable (e.g., because a
% wall was inserted along the path that an agent would normally take),
% but a more ``general'' behavior can be transfered (e.g., I take the
% high road; he takes the low, regardless of the presence or absence of
% obstacles).

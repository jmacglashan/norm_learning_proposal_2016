
\subsubsection*{\large Selecting a Selfish or Social Stance}
\label{sec:stance}

We expect our algorithm as described thus far to be able to interact
with other agents or people and infer social utilities that will guide
it to mutually beneficial behavior. In contrast to existing
machine-learning approaches that achieve cooperative behavior by
making strong assumptions about the algorithmic approach adopted by
the other agents~\cite{conitzer07}, our approach coordinates using the
shared history between the agents at a purely behavioral level.

Neverless, there is a key assumption the algorithm is making about
other agents---that they, too, have the goal of adopting joint
goals. In our initial experiments on human--human play in the Hallway
game, we observed that some pairs readily cooperated while some
adopted a competitive or selfish stance toward the interaction. In the
most interesting cases, one player began with a selfish stance, trying
to block the other agent and get to the goal first, while the other
adopted a social stance, creating opportunities for both players to
succeed. There are several interesting and important computational
problems that arise when the two agents adopt conflicting stances. We
briefly list these problems and our proposed approach to each:

\begin{enumerate}

\item How can an agent detect that its stance differs from the other?
  We will build on our preliminary work~\cite{kleiman16} that uses
  Bayesian reasoning about observed trajectories to infer whether the
  other agent is selfish or social (competitive or cooperative, in the
  terminology of the paper).
  
\item If the agent adopts a social stance and the other adopts a
  selfish stance, how should it respond? Our analysis of CD
  strategies, described earlier, provides a partial answer---an agent
  using such a strategy encourages the other, even if it is
  selfish---to behave in a mutually beneficial way. In particular, a
  CD strategy allows for cooperation but prevents exploitation, making
  the other player an offer it would be foolish to refuse. If a CD
  strategy is not possible, an agent could use a strategy that, across
  repeated games, provides the same benefits as CD strategies within
  games~\cite{munoz08}.
  
\end{enumerate}

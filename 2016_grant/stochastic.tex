
\vspace{\up}
\paragraph{Stochastic games}

The stochastic games formalism can be viewed as an extension of MDPs
to the multi-agent case~\cite{littman1994markov}. 
%In a stochastic game, each of the agents in the environment make decisions simultaneously at each discrete time step and can only observe the other agents' decisions after all decisions have been made and executed in the environment. 
A \mydef{stochastic game} is defined by the tuple $(I, S, A^I, T,
R^I)$, where $I$ is an index set of agents in the environment; $S$ is
the set of states of the environment; $A^I$ is set of actions for each
of the agents with $A^i$ denoting the action set for agent $i \in I$;
$T(s' \mid s, j)$ is the transition dynamics specifying the
probability of the environment transitioning to state $s' \in S$ when
the {\em joint action} $j \in \times_i A^i$ of all agents is taken in
state $s \in S$; and $R^I$ is a set of reward functions for each agent
with $R^i(s, j, s')$ denoting the the reward received by agent
$i \in I$ when the environment transitions to state $s' \in S$ after
the agents take joint action $j \in \times_i A^i$ in state $s \in S$.

The goal in a stochastic game is to find a joint strategy that
satisfies some solution concept. Different solution concepts for
stochastic games have been explored in the past including minimax,
Nash, correlated, and CoCo equilibria
\cite{GreenwaldHall:03,HuWellman03,Littman01,ZGL:06}. There are
problems with these approaches, however. First, the resulting planners
must solve for game-theoretic equilibria in an inner loop, a problem
which in the case of Nash equilibrium, for example, is notorious for
its computational
intractability~\cite{daskalakis2009complexity}. Second, in the general
case of non-constant sum games, none of the planners that make
reasonable assumptions about agent behavior yield unique joint plans,
and none has solved the ensuing equilibrium selection problem
suitably.

Immediately generalizing from the case of MDPs, the goal of inverse
reinforcement learning in stochastic games is to learn a set of reward
functions for the stochastic game that describe an environment that
would motivate the agents to behave in a way that is consistent with
the observed behavior under some solution concept such as a Nash
equilibrium~\cite{reddy2012inverse}.  This problem, however, is
exceedingly difficult, because the planning that is necessary in the
inner loop of an IRL algorithm is subject to the challenges identified
by the equilibrium planners mentioned above.
%
In this project, we plan to develop IRL algorithms that facilitate
equilibrium selection in a stochastic game.  

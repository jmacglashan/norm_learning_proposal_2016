
\vspace{\up}
\paragraph{Turk Experiments}
\label{sec:human}

We conducted analogous behavioral experiments among humans playing
grid games on Mechanical Turk.
%
%In those experiments as well, we were able to manipulate the utilities
%to favor fairness, and doing so induced more cooperation than otherwise.
%
The results are consistent with our simulations of \Q-learners playing
grid games.  In both cases, more collaboration was achieved when the
treatment incorporated social preferences.

While the subjective utility functions of artificial agents are within
our control, so that we can perhaps lead artificial agents towards
collaborative behavior, the subjective utility functions of humans are
not.  Nonetheless, behavioral economists often infer approximations of
utility functions from experimental data by assuming some underlying
structural form, and then estimating the relevant parameters
(e.g.,~\cite{RePEc:eee:ecochp:6a-64}).  Likewise, one of the primary
intended uses of our Turk experimental framework is to generate trace
data of humans playing grid games and learning collaborative behavior
(such as trust, CD, etc.), so that we can then proceed to infer
utility functions and relate them back to the behaviors they produce.


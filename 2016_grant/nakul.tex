
\paragraph{Nakul: Experiments}

In this experiment we want to test our labelled inverse reinforcement
learning algorithm with simple experiments to see if the agent learns
good behavior based on human feedback.  The agent is in a world with
puddles of multiple colors - red, blue, green, yellow and pink. The
agent is allowed north, south, east and west actions.  If there is a
wall in the way of the agent it does not move at all. For example if
the agent tries to move east and it is next to the eastern wall of the
grid, it doesn't move at all.  For the experiments we will demonstrate
two trajectories to the agent, and a pair of labels corresponding to
each trajectory.  We will show that changes in the labels for the same
trajectory lead to changes in the behaviour of the agent.

The first trajectory has the agent move two steps, over the blue
puddle to reach the red puddle.  The second trajectory has the agent
move a single step to the blue puddle.  These trajectories are shown
in the first set of Figures~\ref{fig:trajectories}.

For the first experiment we teach the agent that both behaviours shown
with the demonstrated trajectories are good, that is, walking over
both puddles is good.  Hence, we give a label of $+1$ and $+1$ to both
trajectories. The policy after learning is shown in
Figure~\ref{fig:test1}.  It is clear that the agent prefers to reach
the red puddle with the shortest path and stay over it infinitely. The
agent doesn't learn to stay over the blue puddle as one of the
trajectories was to walk over the blue to reach the red puddle as the
goal state. The agent also values the red puddle higher than any cell
in the grid.

In the second experiment we teach the agent that walking over blue
paddles is bad and getting to the red puddle is good. Hence, we label
the first trajecotry $+1$, and label the second trajectory $-1$.  The
output policy in Figure~\ref{fig:test2} shows that the agent avoids
all blue puddle and gets to the red puddle as quickly as
possible. Further, the red puddle has the highest value in the grid.

In the third experiment we want to teach the agent that the red puddle
should be avoided and the getting to the blue puddle is good. Hence,
we label the first trajectory $-1$ and the second trajectory $+1$.
The output policy in Figure~\ref{fig:test3} shows that the agent
prefers reaching any of the blue puddles in the grid, especially the
ones where it can stay in the blue cells indefinitely.

In the fourth experiment we teach the agent that both red and blue
puddles are bad. Hence, we label both trajectories $-1$.  The output
policy in Figure~\ref{fig:test4} shows that the agent avoids all the
puddles and tries to stay in the white spaces on the grid.

Thus, we can change the agent's behaviour with example trajectories
and more importantly the labels. This learning behaviour is important
for teaching an agent to play with a human, we the human can critique
the agent as they play and lable a round of play good or bad depending
on whether the agents reached their co-operative goal. Further, our
algorithm allows the teacher to critique individual actions that the
agent takes allowing a more minute control of the agent's behaviour.

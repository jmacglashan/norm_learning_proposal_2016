
%\vspace{\up}
\paragraph{Turk Experiments}
\label{sec:human}

We ran two studies in which human subjects played grid games.  We
recruited participants on Mechanical Turk to play the Hallway
game 
%(Figure~\ref{fig:hallway}) 
against another Turker.  

Each participant began with an instruction phase that used a series of
practice grids to teach them the rules of the game: arrow keys to move
north, south, east, or west in the grid; spacebar to wait; both agents
move simultaneously; when two agents try to enter the same square,
their moves fail; and the round ends when either agent reaches a goal.
Example grids demonstrated outcomes in which both agents reached a
goal, and outcomes in which one did and the other did not.  All
transitions (including transitions that did not involve changing
location) were animated so that participants could see that their
actions registered.

%An example of the instruction phase can be viewed at \url{http://goo.gl/SWme3n}. 

After the instruction phase, the participants were paired up.  Each
pair played a match consisting of 20 rounds, which ended when either
or both agents reached a goal, or when they had taken 30 actions
without either reaching a goal.

We designed two different treatments, one (the ``individual''
treatment) was a control, and the other (the ``team'' treatment) was
intended to inspire team reasoning.  In the individual treatment, a
participant received a bonus when they reached their goal, regardless
of whether the other participant also reached their goal.  In the team
treatment, a participant received a bonus only if they reached
their goal at the same time as the other member of their pair.

We recruited 40 participants to form 20 pairs in the individual
treatment; in the team treatment, we recruited 50 participants.  All
participants received \$2.00 as a base payment, and \$0.10 bonuses for
goals scored according to the rules of the treatment.


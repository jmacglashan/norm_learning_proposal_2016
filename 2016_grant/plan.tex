
\vspace{\up}
\paragraph{Plan}

The three-year timeline for our proposed project is as follows:

\begin{itemize}
\item {\bf Year 1}: Demonstrate that a socially rational agent can
  successfully learn collaborative behavior from batches of
  demonstrations (offline), and can generalizate across related games.
  Carry out simulation experiments on machine--machine pairings,
  varying both the algorithms and the environments.  Complete the
  development of an experimental test bed that pairs humans with
  artificial agents.
%
Evaluate algorithms that classify environments as selfish or social,
and adopt the corresponding stance.

%    determine whether an interaction is proceeding collaboratively, 
%    or if cooperation has broken down and players should revert to 
%    more traditional self-interested behavior. 

\item {\bf Year 2}: Establish that IRL algorithms within the GIRL
  framework can learn more accurately and more efficiently (i.e., from
  fewer data) than classic IRL algorithms, because of the availability
  of both positive and negative feedback.  Incorporate these
  algorithms into the Year~1 experimental test bed, and show
  collaborative behavior emerge in real-time during human--machine
  interactions.  Develop the proposed ``Social autonomous driving''
  course (see Broader Impacts).  

\item {\bf Year 3}: Extend as necessary, and then evaluate our
  approach in real-world applications, specifically a ``go fetch''
  scenario to be developed on top of a mobile-manipulation platform
  that is scheduled for deployment in the Brown Computer Science
  department in Spring 2017.

%   Algorithmically, we will focus on ways an agent can actively teach 
%   its partner to adopt mutually beneficial behavior.

\end{itemize}


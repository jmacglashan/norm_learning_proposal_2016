
\vspace{\up}
\paragraph{Plan}

The three-year timeline for our proposed project is as follows:

\begin{itemize}
\item {\bf Year 1}: Demonstrate that a socially rational agent can
  successfully learn collaborative behavior from batches of
  demonstrations (offline), and can generalizate across related games.
  Carry out simulation experiments on machine--machine pairings,
  varying both the algorithms and the environments.  Complete the
  development of an experimental test bed that pairs humans with
  artificial agents.

\item {\bf Year 2}: Establish that IRL algorithms within the GIRL
  framework can learn more accurately and more efficiently (i.e., from
  fewer data) than classic IRL algorithms, because of the ability of
  the teacher to offer both positive and negative feedback.
  Incorporate GIRL learning algorithms into the Year 1 experimental
  test bed, and show how collaborative behavior can emerge in
  real-time during human--machine interactions.  Develop the proposed
  ``Social autonomous driving'' course (see Broader Impacts).
%   determine whether an interaction is proceeding collaboratively, 
%   or if cooperation has broken down and players should revert to 
%   more traditional self-interested behavior. 

\item {\bf Year 3}: Extend as necessary, and then evaluate our
  approach in real-world applications, specifically to the ``go
  fetch'' scenario (see Broader Impacts).
%   Algorithmically, we will focus on ways an agent can actively teach 
%   its partner to adopt mutually beneficial behavior.
%\commentm{We could propose ``deep IRL'' ... since MLIRL is a gradient method, it should be easy to incorporate it into standard deep learning packages to learn complex social utility functions.}

\end{itemize}



\subsubsection*{\large Learning Social Utility Functions}
\label{sec:learning}

% zzz I wasn't surew what ``collaborative learning'' was, so I changed it. I hope that's ok.

Given our representation of social preferences by a social utility
function that is parameterized by a family of bias functions, our
approach to learning these functions is to optimize those parameters
to best incentivize observed behavior.  We consider two learning
frameworks.  In \mydef{batch} learning, an agent learns offline from
demonstrations of other agents playing collaboratively.
In \mydef{interactive} learning, agents learn the social preferences
of the collective in the context of repeated interactions.
%
At the core of both algorithms is team reasoning.

%We consider learning norms under two different conditions. First, when
%an agent observes example behaviors of other agents conforming to a
%norm, and also when an agent must interact with a set of unknown
%agents and develop new norms that facilitate coordination. We refer to
%the first learning situation as {\em batch} norm learning, since the
%agent will receive batches of demonstrations and can perform norm
%learning offline from them; we refer to the latter situation as {\em
%  interactive} norm learning, since the agent must learn norms with
%other agents while interacting with them.%

%To perform both batch and interactive norm learning, we take an
%approach in which the agent reasons about the multi-agent interaction
%as a joint task to solve and then learns biases for behavior in the
%joint task that when coupled with the overall joint task goal, results
%in norm-following behavior. To formalize and solve this learning
%problem, we build from Markov decision process and stochastic games
%formalisms, and inverse reinforcement learning literature. We first
%describe relevant background material on these topics and then
%describe our norm-learning algorithms.



\section{Norm Learning}
\label{sec:learning}

Given our representation of norms by a social reward function that is
parameterized by a family of bias functions, our approach to learning
norms is simply to optimize those parameters to best match observed
behavior.  We consider two forms of norm learning.  The first is
\mydef{batch} learning, in which an agent learns offline from batches
of demonstrations of other agents conforming to a norm.  The second is
\mydef{interactive} learning in which an agent learns about the
behavior of a set of unknown agents, while at the same time
interacting with them and establishing norms.

%We consider learning norms under two different conditions. First, when
%an agent observes example behaviors of other agents conforming to a
%norm, and also when an agent must interact with a set of unknown
%agents and develop new norms that facilitate coordination. We refer to
%the first learning situation as {\em batch} norm learning, since the
%agent will receive batches of demonstrations and can perform norm
%learning offline from them; we refer to the latter situation as {\em
%  interactive} norm learning, since the agent must learn norms with
%other agents while interacting with them.%

%To perform both batch and interactive norm learning, we take an
%approach in which the agent reasons about the multi-agent interaction
%as a joint task to solve and then learns biases for behavior in the
%joint task that when coupled with the overall joint task goal, results
%in norm-following behavior. To formalize and solve this learning
%problem, we build from Markov decision process and stochastic games
%formalisms, and inverse reinforcement learning literature. We first
%describe relevant background material on these topics and then
%describe our norm-learning algorithms.



\subsection{Mathematical Models}
\label{sec:models}

As grid games are examples of stochastic
games~\cite{Fink64,Shapley53}, we use this game-theoretic
framework to model agent interactions.
%
Like econometricians~\cite{RePEc:eee:ecochp:6a-64}, we assume a
structural form for utility functions; and because we are interested
in building socially rational artificial agents, we endow our agents
with a (potentially) social utility function.
%
We then develop a batch learning algorithm by which an artificial
agent can, by learning from demonstrations, recover a social utility
function that supports exactly one of many equilibria (assuming the
demonstrations exhibit exactly one of many equilibria).
%
We also develop an interactive learning algorithm by which two agents
can simultaneously learn a social utility function, and converge
%(very quickly)
to one among a plethora of equilibria, thereby endogenously solving
the equilibrium selection-process.

\comment{
Both of these algorithms rely on IRL, and in our experiments thus far
they invoke only standard IRL (e.g.,~\cite{MLIRL, BIRL}).
As standard IRL cannot accommodate negative feedback,
% \commenta{BLAH BLAH BLAH}
}


\vspace{\up}
\paragraph{Markov decision processes}

A \mydef{Markov Decision Process} (MDP) is a model of a single-agent
decision-making problem defined by the tuple $(S, A, T, R)$, where
$S$ is the set of states in the world; $A$ is the set of actions that
the agent can take; $T(s' \mid s, a)$ defines the transition dynamics:
the probability the environment transitions to state $s' \in S$
after the agent takes action $a \in A$ in state $s \in S$; and 
$R(s, a, s')$ is the utility (or reward) function, which returns the reward the
agent receives when environment transitions to state $s'$ after the
agent takes action $a$ in state $s$.

The goal of planning or learning in an MDP is to find a policy $\pi :
S \rightarrow A$ (a mapping from states to actions) that maximizes the
expected future discounted reward under that policy: $E^{\pi} \left[
  \sum_{t=0}^\infty \gamma^t R(s_t, a_t, s_{t+1}) \right]$, where
$\gamma \in [0, 1]$ is a discount factor specifying how much immediate
rewards are favored over distant rewards. 
%Since this sum goes to the infinite future, this objective is called an {\em infinite horizon} objective.

To find an optimal policy, many algorithms compute the optimal state
($V^*(s)$) and state--action ($Q^*(s,a)$) value functions that specify
the expected future discount reward under the optimal policy from each
state, and from taking an action in a state and then following the
optimal policy respectively. 
%These functions are defined recursively
%by the Bellman equations:
%%
%\begin{equation}
%V^*(s) = \max_{a \in A} \sum_{s' \in S} T(s' \mid s, a) \left[ R(s, a, s') + \gamma V^*(s') \right].
%\end{equation}
%and
%\begin{equation}
%Q^*(s,a) = \sum_{s' \in S} T(s' \mid s, a) \left[ R(s, a, s') + \gamma V^*(s') \right].
%\end{equation}
%\noindent
Given these functions, the optimal policy is derived by taking an
action in each state with the maximum Q-value: 
$\pi(s) \in \arg\max_{a \in A} Q(s, a)$~\cite{bertsekas87}.



\vspace{\up}
\paragraph{Inverse reinforcement learning}

Although there are multiple IRL formalizations and approaches, they
all take as input an MDP together with a \mydef{family} of a reward
functions $R_\Theta$ that is defined by some parameter space $\Theta$,
and a dataset $D$ of trajectories (where a trajectory is a finite
sequence of state-action pairs: $\langle (s_1, a_1), ..., (s_n, a_n)
\rangle$). The algorithms then output a specific parameterized reward
function $R_\theta \in R_\Theta$, which induces a policy that is
consistent with the input trajectories.
%
Different IRL algorithms frame the objective function for policy
consistency differently. One common approach is to treat the
search problem as a probabilistic inference
problem~\cite{babes11,lopes2009active,ramachandran2007bayesian,ziebart2008maximum}. 

For example, in the maximum-likelihood setting~\cite{babes11}, the
goal is to find a reward function parameterization that maximizes
the likelihood of the data:
%
\begin{equation}
\label{eq:mlirl}
\theta \in \arg\max_{\theta} L(D \mid R_{\theta}) = \prod_{t \in D} \prod_i^{|t|} \pi_{\theta}(s_i, a_i),
\end{equation}

\noindent
where $\pi_{\theta}(s, a)$ is a stochastic policy defining the
probability of taking action $a$ in state $s$ when the parameterized
reward function to be maximized is $R_{\theta}$. Typically, the
Boltzmann (softmax) stochastic policy over the $Q$-values is used:
$\pi_{\theta}(s, a) = \frac{e^{\beta Q_{\theta}(s,a)}}{\sum_{a' \in A}
  e^{\beta Q_{\theta}(s,a')}}$, where $Q_\theta(s, a)$ is the $Q$-function
when the reward function is parameterized by $\theta$.
%
Popular methods for solving Equation~\ref{eq:mlirl} (i.e., for
carrying out maximum-likelihood IRL (MLIRL)) include gradient ascent
and expectation-maximization.
%\commenta{do these algorithms need references?}

In Bayesian IRL, the goal is to compute a posterior distribution over
reward function parameterizations:
%
%\begin{equation}
%\label{eq:birl}
$\Pr(R_{\theta} \mid D) \propto \Pr(D \mid R_{\theta}) \Pr(R_\theta)$.
%\end{equation}
%
Bayesian IRL algoriths typically rely on Markov chain Monte Carlo
methods~\cite{journals/corr/abs-1208-2112,ramachandran2007bayesian}.
%
Other approaches include maximum-entropy~\cite{ziebart2008maximum}
and relative entropy IRL~\cite{BoulariasKP2011}. 
%
More recently, work has also commenced on
nonparametric Bayesian IRL methods~\cite{Choi:2012:NBI:2999134.2999169,Michini2012}.

IRL algorithms are typically computationally demanding because they
require planning in their inner loops.
%If the social reward function learning is going to be used in any
%moderately interactive setting, this limitation may be prohibitive. 
%
Compatible with all of the above algorithms, \mydef{receding horizon
IRL} (RHIRL)~\cite{macglashan15b} addresses this limitation by
replacing the usual infinite-horizon policy with a receding horizon
controller (RHC) that only plans out to some finite horizon from any
given state, thereby bounding computation time. An RHC tends to steer
an IRL algorithm towards a \mydef{shaping reward function}~\cite{Ng:1999:PIU:645528.657613}
%https://people.eecs.berkeley.edu/~pabbeel/cs287-fa09/readings/NgHaradaRussell-shaping-ICML1999.pdf
short-term rewards that guide the agent without it having to plan too
far ahead.  Consequently, RHIRL allows us to plan with a short
horizon, thereby saving on computation time.

Our computational framework, and the generalized model of IRL we
introduce in Section~\ref{sec:girl}, are all agnostic about the
particular approach to IRL taken.  We plan to experiment with multiple
variations.




\vspace{\up}
\paragraph{Stochastic games}

The stochastic games formalism can be viewed as an extension of MDPs
to the multi-agent case~\cite{littman1994markov}. 
%In a stochastic game, each of the agents in the environment make decisions simultaneously at each discrete time step and can only observe the other agents' decisions after all decisions have been made and executed in the environment. 
A \mydef{stochastic game} is defined by the tuple $(I, S, A^I, T,
R^I)$, where $I$ is an index set of agents in the environment; $S$ is
the set of states of the environment; $A^I$ is set of actions for each
of the agents with $A^i$ denoting the action set for agent $i \in I$;
$T(s' \mid s, j)$ is the transition dynamics specifying the
probability of the environment transitioning to state $s' \in S$ when
the {\em joint action} $j \in \times_i A^i$ of all agents is taken in
state $s \in S$; and $R^I$ is a set of reward functions for each agent
with $R^i(s, j, s')$ denoting the the reward received by agent
$i \in I$ when the environment transitions to state $s' \in S$ after
the agents take joint action $j \in \times_i A^i$ in state $s \in S$.

The goal in a stochastic game is to find a joint strategy that
satisfies some solution concept. Different solution concepts for
stochastic games have been explored in the past including minimax,
Nash, correlated, and CoCo equilibria
\cite{GreenwaldHall:03,HuWellman03,Littman01,ZGL:06}. There are
problems with these approaches, however. First, the resulting planners
must solve for game-theoretic equilibria in an inner loop, a problem
that, in the case of Nash equilibrium, for example, is notorious for
its computational
intractability~\cite{daskalakis2009complexity}. Second, in the general
case of non-constant-sum games, none of the planners that make
reasonable assumptions about agent behavior yield unique joint plans,
and none has solved the ensuing equilibrium-selection problem
suitably.

Immediately generalizing from the case of MDPs, the goal of inverse
reinforcement learning in stochastic games is to learn a set of reward
functions for the stochastic game that describe an environment that
would motivate the agents to behave in a way that is consistent with
the observed behavior under some solution concept such as a Nash
equilibrium~\cite{reddy2012inverse}.  This problem, however, is
exceedingly difficult, because the planning that is necessary in the
inner loop of an IRL algorithm is subject to the challenges identified
by the equilibrium planners mentioned above.
%
In this project, we plan to develop IRL algorithms that facilitate
equilibrium selection in a stochastic game by leveraging shared experience.


\vspace{\up}
\paragraph{Social Utility Functions}

In our preliminary studies, we assume players' individual (game)
utilities are known, and our goal is to recover a \mydef{social utility
  function} that combines these known utilities in such a way as to
capture social preferences expressed in the joint play of the agents
(when one exists).

Like the players' reward functions, a social utility function operates
on states, joint actions, and next states: $R^S : S \times_i A^i
\times S$.  In our initial model, we assume the social utility function
can be written as a linear combination of a \mydef{team function}, which
represents team goals,
%an \mydef{individual function}, which represents an individual's goals,
and a family of what we call \mydef{bias functions} ($B_\Theta$)
defined by some parameter space $\Theta$.  The team function takes as
input a multi-agent utility function ($R^I)$, and returns a single
numeric ``team'' value for any state, joint action, next state
triple. One example of such a function is total welfare (i.e., the sum
of all agent utilities): ${\mathcal T}(R^I, s, j, s') = \sum_i R^i(s, j, s')$.
%
The bias function family is similar in nature to a utility function
family that would be input to a classic IRL algorithm, but operates on
state, joint action, next state triples, thereby encoding a bias that
motivates specific collaborative behavior in games.  In addition to the
parameters of the bias function, the social utility function may also
include a parameter
%$\alpha$ 
to trade-off the team utilities against the biases.
%$R^S(s, j, s') = \alpha {\mathcal T}(R^I, s, j, s') + (1-\alpha) B_\Theta(s, j, s')$.
%but in the simplest case, $\alpha = 0.5$.

%A viable alternative might be to design the bias functions such that
%agents can receive a bonus as they move past each other on the right.
%Motivating the desired behavior in this way could result in greater 
%generalization power.  To facilitate generalization, the bias function 
%family must operate on a set of useful features.  



\comment{
WHY INTENTION LEARNING BEATS SUPERVISED METHODS!

Recall our working definition of a social norm as a behavioral
instruction that members of the community expect one another to
follow.  This definition suggests that norms could potentially be
learned in a supervised fashion simply by learning a function that
maps states to joint actions directly, rather than capturing norms in
a social reward function that motivates joint actions.
%\commenta{do you mean just hard code a norm in the bias functions? 
%does something like this next sentence capture what you mean?}
%\commenta{This definition suggests that norms could be directly encoded in the bias functions of a social reward function as bonuses associated with specific joint actions.}
However, as an environment and a target policy become more complex,
directly learning the policy becomes more challenging, and
generalizing from it likely less successful.  As with standard IRL,
the advantage of learning a social reward function instead is that
simple reward functions can often induce complex
behaviors \emph{across states}.  For example, consider a seemingly
straightforward norm that two agents approaching one another each stay
to their right.  If the action space is over low-level controls (e.g.,
rotations and small movements) and the policy is context dependent
(e.g., navigating obstacles to get to the right side), implementing
all the joint action rules necessary to make the agents pass on the
right could be difficult to specify (and hence, learn).  However, a
social reward function can simply define a bonus for when the agents
move past each other on the right and the planning algorithm does the
rest.
}




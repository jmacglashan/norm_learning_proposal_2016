
\section{Norm Representation}
\label{sec:representation}

We use the framework of stochastic games to model agent interactions.
Because we are operating under the assumption that human behavior is
social, we assume a (potentially) social reward function.  We then
develop an interactive learning algorithm by which two machines can
converge (very quickly) to one among a plethora of equilibria.  We
also develop a batch learning algorithm by which a machine can learn
from demonstrations a social reward function that supports one of many
equilibria.  Both of these algorithms rely on IRL, and in our
experiments thus far they invoke only standard IRL (e.g.,~\cite{MLIRL, BIRL}).

As standard IRL cannot accommodate negative feedback,
\commenta{BLAH BLAH BLAH}

\paragraph{Markov decision processes}

A \mydef{Markov Decision Process} (MDP) is a model of a single-agent
decision making problem, defined by the tuple $(S, A, T, R)$, where
$S$ is the set of states in the world; $A$ is the set of actions that
the agent can take; $T(s' \mid s, a)$ defines the transition dynamics:
the probability the environment transitions to state $s' \in S$
after the agent takes action $a \in A$ in state $s \in S$; and 
$R(s, a, s')$ is the reward function, which returns the reward the
agent receives when environment transitions to state $s'$ after the
agent takes action $a$ in state $s$.

The goal of planning or learning in an MDP is to find a policy $\pi :
S \rightarrow A$ (a mapping from states to actions) that maximizes the
expected future discounted reward under that policy: $E^{\pi} \left[
  \sum_{t=0}^\infty \gamma^t R(s_t, a_t, s_{t+1}) \right]$, where
$\gamma \in [0, 1]$ is a discount factor specifying how much immediate
rewards are favored over distant rewards. 
%Since this sum goes to the infinite future, this objective is called an {\em infinite horizon} objective.

To find an optimal policy, many algorithms compute the optimal state
($V^*(s)$) and state-action ($Q^*(s,a)$) value functions that specify
the expected future discount reward under the optimal policy from each
state, and from taking an action in a state and then following the
optimal policy respectively. 
%These functions are defined recursively
%by the Bellman equations:
%%
%\begin{equation}
%V^*(s) = \max_{a \in A} \sum_{s' \in S} T(s' \mid s, a) \left[ R(s, a, s') + \gamma V^*(s') \right].
%\end{equation}
%and
%\begin{equation}
%Q^*(s,a) = \sum_{s' \in S} T(s' \mid s, a) \left[ R(s, a, s') + \gamma V^*(s') \right].
%\end{equation}
%\noindent
Given these functions, the optimal policy is derived by taking an
action in each state with the maximum Q-value: 
$\pi(s) \in \arg\max_{a \in A} Q(s, a)$~\cite{bertsekas87}.

\paragraph{Stochastic games}

The stochastic games formalism can be viewed as an extension of MDPs
to the multi-agent case~\cite{littman1994markov}. 
%In a stochastic game, each of the agents in the environment make decisions simultaneously at each discrete time step and can only observe the other agents' decisions after all decisions have been made and executed in the environment. 
A \mydef{stochastic game} is defined by the tuple $(I, S, A^I, T,
R^I)$, where $I$ is an index set of agents in the environment; $S$ is
the set of states of the environment; $A^I$ is set of actions for each
of the agents with $A^i$ denoting the action set for agent $i \in I$;
$T(s' \mid s, j)$ is the transition dynamics specifying the
probability of the environment transitioning to state $s' \in S$ when
the {\em joint action} $j \in \times_i A^i$ of all agents is taken in
state $s \in S$; and $R^I$ is a set of reward functions for each agent
with $R^i(s, j, s')$ denoting the the reward received by agent
$i \in I$ when the environment transitions to state $s' \in S$ after
the agents take joint action $j \in \times_i A^i$ in state $s \in S$.

The goal in a stochastic game is to find a joint strategy that
satisfies some solution concept. Different solution concepts for
stochastic games have been explored in the past including minimax,
Nash, correlated, and CoCo equilibria
\cite{GreenwaldHall:03,HuWellman03,Littman01,ZGL:06}. There are
problems with these approaches, however. First, the resulting planners
must solve for game-theoretic equilibria in an inner loop, a problem
which in the case of Nash equilibrium, for example, is notorious for
its computational intractability~\cite{daskalakis2009complexity}. Second, in the
general case of non-constant sum games, none of the planners that make
reasonable assumptions about agent behavior yield unique joint plans,
and furthermore, none has solved the ensuing equilibrium selection
problem suitably.

Immediately generalizing from the case of MDPs, the goal of inverse
reinforcement learning in stochastic games is to learn a set of reward
functions for the stochastic game that describe an environment that
would motivate the agents to behave in a way that is consistent with
the observed behavior under some solution concept such as a Nash
equilibrium~\cite{reddy2012inverse}.  This problem, however, is
exceedingly difficult, because the planning that is necessary in the
inner loop of an IRL algorithm is subject to the challenges identified
by the equilibrium planners mentioned above.

\commenta{JuNK!!!}
%This social reward function is our norm representation.

In this project, we plan to develop IRL algorithms that support
equilibrium selection in a stochastic game.  In our preliminary
studies, we assume players' rewards are known, and our goal is to
recover a \mydef{social reward function} that combines these known
rewards in such a way as to capture a social norm expressed in the
joint play of the agents (when one exists).

Like the players' reward functions, a social reward function operates
on states, joint actions, and next states: $R^S : S \times_i A^i
\times S$.  In our initial model, we assume that this social reward
function can be written as linear combination of a \mydef{team
  function}, which represents team goals,
%an \mydef{individual function}, which represents an individual's goals,
and a family of what we call \mydef{bias functions} ($B_\Theta$)
defined by some parameter space $\Theta$.  The team function takes as
input a multi-agent reward function ($R^I)$, and returns a single
numeric value for any state-joint action-next state triple that
represents a team goal. One example of such a function is total
welfare (i.e., the sum of all agent rewards): 
${\mathcal T}(R^I, s, j, s') = \sum_i R^i(s, j, s')$.
%
The bias function family is similar in nature to a reward function
family that would be input to a classic IRL algorithm, but operates on
state-joint action-next state triples, thereby encoding a bias that
motivates norm-adhering behavior in games.  In addition to the
parameters of the bias function, the social reward function may also
include a parameter
%$\alpha$ 
to trade-off the team rewards against the biases.
%$R^S(s, j, s') = \alpha {\mathcal T}(R^I, s, j, s') + (1-\alpha) B_\Theta(s, j, s')$.
%but in the simplest case, $\alpha = 0.5$.

\comment{
\commenta{The individual function \ldots} \jmnote{Do we want to add
  an individual function here? Right now the social reward function
  in the pseudocode (and results) do not include an individual term.
  It also makes the batch mode less straight forward because
  it means there would be a different social reward function
  for each agent. Would that mean we run batch once for each agent?
  I'm going to comment it out for now, but we can add it back.}
}

%\jmnote{Not sure this whole above paragraph that I added is needed here; maybe best to incorporate the ideas earlier, but I was trying to bridge to why we care about features}

Recall our working definition of a social norm as a behavioral
instruction that members of the community expect one another to
follow.  This definition suggests that norms could potentially be
learned in a supervised fashion simply by learning a function that
maps states to joint actions directly, rather than capturing norms in
a social reward function that motivates joint actions.
%\commenta{do you mean just hard code a norm in the bias functions? 
%does something like this next sentence capture what you mean?}
%\commenta{This definition suggests that norms could be directly encoded in the bias functions of a social reward function as bonuses associated with specific joint actions.}
However, as an environment and a target policy become more complex,
directly learning the policy becomes more challenging, and
generalizing from it likely less successful.  As with standard IRL,
the advantage of learning a social reward function instead is that
simple reward functions can often induce complex
behaviors \emph{across states}.  For example, consider a seemingly
straightforward norm that two agents approaching one another each stay
to their right.  If the action space is over low-level controls (e.g.,
rotations and small movements) and the policy is context dependent
(e.g., navigating obstacles to get to the right side), implementing
all the joint action rules necessary to make the agents pass on the
right could be difficult to specify (and hence, learn).  However, a
social reward function can simply define a bonus for when the agents
move past each other on the right and the planning algorithm does the
rest.

%\commenta{this is a seaparte point}
%trade offs between the team goal and biases for norms can be learned. 

To facilitate social reward functions that generalize, the bias
function family must operate on a set of useful features.  As part of
this project, we plan to investigate whether existing state-of-the-art
feature selection methods for
RL~\cite{diuk2009adaptive,kolter2009regularization,li2009reinforcement,parr2008analysis}
work well in our setting, and then to work towards developing new
methods to the extent necessary.

%A viable alternative might be to design the bias functions such that
%agents can receive a bonus as they move past each other on the right.
%The intent is that motivating the desired behavior in this way would
%result in greater generalization power.  To facilitate this kind of
%generalization, the bias function family must operate on a set of
%useful features.  As part of this project, we plan to investigate
%whether existing state-of-the-art feature selection methods for
%RL~\cite{diuk2009adaptive,kolter2009regularization,li2009reinforcement,parr2008analysis}
%work well in our setting, and then to work towards developing new
%methods to the extent necessary.

%\jmnote{I made a bit of a bold claim here saying that we would investigate feature selection. I say that because this might indeed be an interesting space that benefits from new kinds of features, but if we don't want to commit to this, we can just say its outside the scope of our work.}

\comment{
We believe that the choice to represent norms by a single social
reward function is grounded in sound psychology because\commenta{JOE!!??}.
%
Furthermore, this assumption is essential for our approach to be
tractable, because IRL in MDPs is becoming increasingly easier, while
IRL in stochastic games remains enormously challenging.
}

